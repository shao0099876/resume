% !TEX TS-program = xelatex
% !TEX encoding = UTF-8 Unicode
% !Mode:: "TeX:UTF-8"

\documentclass{resume}
\usepackage{zh_CN-Adobefonts_external} % Simplified Chinese Support using external fonts (./fonts/zh_CN-Adobe/)
% \usepackage{NotoSansSC_external}
% \usepackage{NotoSerifCJKsc_external}
% \usepackage{zh_CN-Adobefonts_internal} % Simplified Chinese Support using system fonts
\usepackage{linespacing_fix} % disable extra space before next section
\usepackage{cite}

\begin{document}
\pagenumbering{gobble} % suppress displaying page number

\name{邵若忱}

\basicInfo{
  \email{shao0099876@outlook.com} \textperiodcentered\ 
  \phone{(+86) 178-631-07810} \textperiodcentered\ 
  \linkedin[邵若忱]{https://www.linkedin.com/in/%E8%8B%A5%E5%BF%B1-%E9%82%B5-75947b139/}}
 
\section{\faGraduationCap\  教育背景}
\datedsubsection{\textbf{哈尔滨工业大学(威海)}, 威海}{2016 -- 至今}
\textit{本科在读} 计算机科学与技术, 预计 2020 年 7 月毕业
\begin{itemize}
  \item GPA:3.17(排名年级前40%)
  \item 相关课程:算法设计与分析,计算机网络,操作系统,嵌入式系统设计与开发等
\end{itemize}

\section{\faUsers\ 项目与工作经历}
\datedsubsection{\textbf{ACM-ICPC 校队} }{2017年9月 -- 至今}
\role{队长、学生教练}{指导教师:董开坤}
\begin{itemize}
  \item 参与校队参与赛事的现场组织协调工作
  \item 参与校队资源分配工作
  \item 参与校级比赛命题工作与裁判工作
\end{itemize}

\datedsubsection{\textbf{Balmy语言解释器}}{2017年12月 -- 2018年1月}
\role{Java}{个人项目}
\begin{onehalfspacing}
独立设计语言语法及编译算法, https://github.com/shao0099876/Balmy
\begin{itemize}
  \item 编译算法自主思考、自主设计(后在另一项目中升级为主流算法)
  \item 代码容易扩展
  \item 代码着色功能
\end{itemize}
\end{onehalfspacing}

\datedsubsection{\textbf{Edit4droid}}{2018 年1月 -- 2018年2月}
\role{Android, Java}{个人项目}
\begin{onehalfspacing}
仿C4droid 安卓平台C语言编辑器
\begin{itemize}
  \item 代码着色
\end{itemize}
\end{onehalfspacing}

\datedsubsection{\textbf{AmyYoga瑜伽馆网站}}{2019 年3月 -- 2019年6月}
\role{Django, Linux, Python}{团队项目}
\begin{onehalfspacing}
AmyYoga瑜伽馆服务网站
\begin{itemize}
  \item 全栈开发,参与前端设计,后端主力开发,负责服务器运维与部署
  \item 负责项目架构设计,参与需求分析、测试等流程
  \item 负责持续集成、持续部署工具的配置
\end{itemize}
\end{onehalfspacing}

% Reference Test
%\datedsubsection{\textbf{Paper Title\cite{zaharia2012resilient}}}{May. 2015}
%An xxx optimized for xxx\cite{verma2015large}
%\begin{itemize}
%  \item main contribution
%\end{itemize}

\section{\faHeartO\ 获奖情况}
\datedline{\textit{银奖}, ACM-ICPC亚洲区域赛青岛站}{2018 年11 月}
\datedline{\textit{320分,排名全国前1.9\%}, 第15次CCF计算机软件能力认证}{2018 年12 月}

\datedline{\textit{铜奖}, ACM-ICPC亚洲区域赛沈阳站}{2018 年10 月}
\datedline{\textit{银奖}, 山东省大学生ACM程序设计竞赛}{2018 年5 月}
\datedline{\textit{金奖,第二名}, 威海市大学生ACM程序设计竞赛}{2018 年4 月}

\section{\faCogs\ 专业技能}
% increase linespacing [parsep=0.5ex]
\begin{itemize}[parsep=0.5ex]
  \item 编程语言: 应用程序开发主要使用Java,算法开发主要使用C、C++,能够使用Python,了解JavaScript,乐于接受新语言
  \item 硬件技能: 能够使用Verilog进行FPGA编程,了解并实践过STM32单片机开发,有自主独立开发相关产品的想法与计划
  \item 嵌入式系统技能:略微了解嵌入式Linux原理
\end{itemize}



\section{\faInfo\ 其他}
% increase linespacing [parsep=0.5ex]
\begin{itemize}[parsep=0.5ex]
  \item 技术博客: https://www.cnblogs.com/shao0099876
  \item GitHub: https://github.com/shao0099876
  \item ICPCID: https://icpc.baylor.edu/ICPCID/8OYB1BZCCXXC
  \item 语言: 普通话 - 母语,英语 - 熟练(CET-6 510)
\end{itemize}

%% Reference
%\newpage
%\bibliographystyle{IEEETran}
%\bibliography{mycite}
\end{document}
